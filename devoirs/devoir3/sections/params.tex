\documentclass[french,11pt]{exam}

\usepackage[utf8]{inputenc}
\usepackage[T1]{fontenc}
\usepackage{babel}

\usepackage{geometry}
\geometry{margin=0.6in}

\usepackage{amsmath, amsfonts, amsthm, amssymb, mathtools, centernot}
\usepackage[inline]{asymptote}
\usepackage{enumerate, enumitem, listings}
\usepackage{graphicx, wrapfig, hyperref}
\usepackage{booktabs}
\usepackage{caption}
\usepackage{tikz}
\usepackage{tkz-fct}
\usepackage{subcaption}
\usepackage{tikz-feynman}
\usepackage{physics}
\usepackage{multicol}
\usepackage{multirow}
\usepackage{array}
    \newcolumntype{P}[1]{>{\centering\arraybackslash}p{#1}}
    \newcolumntype{M}[1]{>{\centering\arraybackslash}m{#1}}
\usepackage{ dsfont }
%\usepackage[dvipsnames]{xcolor}

\DeclarePairedDelimiter{\ceil}{\lceil}{\rceil}
\DeclarePairedDelimiter{\floor}{\lfloor}{\rfloor}
%\DeclarePairedDelimiter{\abs}{\vert}{\vert}
\DeclarePairedDelimiter{\p}{\left(}{\right)}

\setlength{\parindent}{0pt}
\setlength{\parskip}{5pt plus 1pt}
%\pagestyle{empty}

\newcommand{\questiontype}{Question}
\newcommand{\writtensection}{0}

\def\indented#1{\list{}{}\item[]}
\let\indented=\endlist

\newcounter{questionCounter}
\newcounter{partCounter}[questionCounter]


\newenvironment{namedquestion}[1][\arabic{questionCounter}]{%
    \refstepcounter{questionCounter}
    \addtocounter{questionCounter}{1}%
    \setcounter{partCounter}{0}%
    \vspace{.2in}%
        \noindent{\bf #1}%
    \vspace{0.3em} \hrule \vspace{.1in}%
}{}

\newenvironment{numedquestion}[0]{%
	\stepcounter{questionCounter}%
    \vspace{.2in}%
        \ifx\writtensection\undefined
        \noindent{\bf \questiontype \; \arabic{questionCounter}. }%
        \else
          \if\writtensection0
          \noindent{\bf \questiontype \; \arabic{questionCounter}. }%
          \else
          \noindent{\bf \questiontype \; \writtensection.\arabic{questionCounter} }%
        \fi
    \vspace{0.3em} \hrule \vspace{.1in}%
}{}

\newenvironment{alphaparts}[0]{%
  \begin{enumerate}[label=\textbf{(\alph*)}]
}{\end{enumerate}}

\newenvironment{arabicparts}[0]{%
  \begin{enumerate}[label=\textbf{\arabic{questionCounter}.\arabic*})]
}{\end{enumerate}}

\newenvironment{questionpart}[0]{%
  \item
}{}

\newcommand{\vh}[1]{\boldsymbol{\hat{#1}}}
%\newcommand{\vb}[1]{\mathbf{#1}}

\newcommand{\answerbox}[1]{
\begin{framed}
\vspace{#1}
\end{framed}}







\newcommand{\Noms}{Pierre-Antoine Graham}
\newcommand{\typeDev}{Devoir}
\newcommand{\numeroDev}{3}
\newcommand{\titre}{\typeDev \space \numeroDev}
\newcommand{\nomCours}{Physique subatomique, PHQ638}
\newcommand{\typeDeQuestion}{}
\newcommand{\Date}{14 novembre 2022}


\newcommand{\col}[1]{\begin{pmatrix} #1 \end{pmatrix}}
%\newcommand{\vh}[1]{\mathbf{\hat{#1}}}%vecteur unitaires en gras
\newcommand{\til}{\widetilde}
\newcommand{\To}{\Longrightarrow}
\newcommand{\From}{\Longleftarrow}
\newcommand{\pdvc}[3]{\qty(\pdv{#1}{#2})_{#3}}
\newcommand{\beps}{\boldsymbol\epsilon}


\newcommand{\courtTitre}
{\thispagestyle{plain}
\begin{center}
  {\Large \scshape \nomCours: \typeDev\space \numeroDev} \\
  \Noms{} \\
  \Date

\end{center}

\cfoot{\thepage}
%\vspace{-0.5cm}


}
